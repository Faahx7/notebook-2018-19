\documentclass[12pt]{article}
\usepackage{graphicx}
\begin{document}
\begin{center}
September 22nd - Chris Lonergan
\end{center}

\section{To Start:}
We first went over the basic parts of the robot (Core Motor Controller, Core Power Distribution Module, chassis, ect.) After that, we watched the FTC video on the competition to gauge our objectives this year. 

\section{Specifying Requirements:}
We realized that there were certain parameters that our robot needed to fulfill. 
\begin{itemize}
	\item 18x18x18 inches Cubed
	\item 8 Motors max
	\item 42 lb max
\end{itemize}
\includegraphics[scale=1]{blah.png}%image INSERT HERE of board with stuff

\section{Divide And Conquer:}
Next, we separated our team into four groups: Chassis and Drive Chain, Lifting and Lowering Robot, Autonomous, and Blocks and Balls.

\subsection{Chassis and Drive Chain:}
The Chassis and Drive Chain group decided that the best method was a tank drive, a triangular shape that uses treads instead of wheels. 
\includegraphics[scale=1]{blah.png}%image INSERT HERE
This design would be able to incorporate the drawer slide and shovel ideas from the other groups as well. It would also serve as an easier method for climbing over the sides of the crater, as the treads are built for terrain.

\subsection{Lifting and Lowering Robot:}
The lifting and Lowering Robot group decided on using a drawer slide to lift the robot off of the ground. It will be at the center of gravity so that the robot can lift vertically without falling over. 
\includegraphics[scale=1]{blah.png}%image INSERT HERE

\subsection{Autonomous:}
The autonomous group made a flow chart of the program that the robot will need to follow. It outlined all of the steps we need to take in the Autonomous section and how many points each objective is worth.

\subsection{Blocks and Balls:}
An important part of the mission is gathering materials and moving them around. The group had several ideas.
\begin{itemize}
	\item Launching 
	\item Arm/Claw
	\item Shovel
\end{itemize}
The idea of launching balls was quickly discarded, as it is inaccurate and could be unsafe. The arm/claw idea was designed for the square minerals, as if we put 4 'fingers' on the arms it could grab the square by the edges and firmly grasp it. The other idea was a shovel. The shovel would be thin and scoop up the materials, but only have room for two materials, so that we do not accidentally pick up three at once, which would be against the rules.
\includegraphics[scale=1]{blah.png}%image INSERT HERE

\section{Wrapping Up:}
To end our practice, all the groups came together and shared their ideas and designs with everyone else. We set up a schedule for practices and brainstormed as a group to fine tune several of our ideas. The notebook group also tried to lay out and organize a system for the notebook, and introduced the newer members to LaTex, the programming language we are using for notebook.
\includegraphics[scale=1]{blah.png}%image INSERT HERE

\end{document}